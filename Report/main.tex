%%%%%%%%%%%%%%%%%%%%%%%%%%%%%%%%%%%%%%%%%%%%%%%%%%%%%%%%%%%%%%%%%%%%%%
% writeLaTeX Example: Academic Paper Template
%
% Source: http://www.writelatex.com
% 
% Feel free to distribute this example, but please keep the referral
% to writelatex.com
% 
%%%%%%%%%%%%%%%%%%%%%%%%%%%%%%%%%%%%%%%%%%%%%%%%%%%%%%%%%%%%%%%%%%%%%%
% How to use writeLaTeX: 
%
% You edit the source code here on the left, and the preview on the
% right shows you the result within a few seconds.
%
% Bookmark this page and share the URL with your co-authors. They can
% edit at the same time!
%
% You can upload figures, bibliographies, custom classes and
% styles using the files menu.
%
% If you're new to LaTeX, the wikibook is a great place to start:
% http://en.wikibooks.org/wiki/LaTeX
%
%%%%%%%%%%%%%%%%%%%%%%%%%%%%%%%%%%%%%%%%%%%%%%%%%%%%%%%%%%%%%%%%%%%%%%
\documentclass[twocolumn,showpacs,%
  nofootinbib,aps,superscriptaddress,%
  eqsecnum,prd,notitlepage,showkeys,10pt]{revtex4-1}

% \usepackage{biblatex}
% \addbibresource{Report/referencias.bib}
% \addbibresource{Report/mainNotes.bib}

\usepackage[portuguese]{babel}
\usepackage[shortlabels]{enumitem}
\usepackage[utf8]{inputenc}
\usepackage{amssymb}
\usepackage{amsmath}
\usepackage{graphicx}
\usepackage{dcolumn}
\usepackage{xcolor}
\usepackage[pdfstartview=FitH,
            colorlinks,
            bookmarksnumbered,
            bookmarksopen,
            linktocpage,
            urlcolor=blue,
            linkcolor=red!70!black,
            citecolor=red!70!black]{hyperref}
\usepackage{float}
% \usepackage{multicol}
\usepackage{listings}
\usepackage{xfrac}
\usepackage{color}
\usepackage{tikz}
\usetikzlibrary{shapes, arrows}

\tikzstyle{terminator} = [rectangle, draw, text centered, rounded corners, minimum height=2em]
\tikzstyle{process} = [rectangle, draw, text centered, minimum height=2em]
\tikzstyle{decision} = [diamond, draw, text centered, minimum height=2em]
\tikzstyle{data}=[trapezium, draw, text centered, trapezium left angle=60, trapezium right angle=120, minimum height=2em]
\tikzstyle{connector} = [draw, -latex']

\definecolor{dkgreen}{rgb}{0,0.6,0}
\definecolor{gray}{rgb}{0.5,0.5,0.5}
\definecolor{mauve}{rgb}{0.58,0,0.82}

\lstset{frame=tb,
  language=[90]Fortran,
  aboveskip=3mm,
  belowskip=3mm,
  showstringspaces=false,
  columns=flexible,
  basicstyle={\small\ttfamily},
  numbers=none,
  numberstyle=\tiny\color{gray},
  keywordstyle=\color{blue},
  commentstyle=\color{dkgreen},
  stringstyle=\color{mauve},
  breaklines=true,
  breakatwhitespace=true,
  tabsize=4
}

\usepackage{booktabs}
\setlength{\heavyrulewidth}{1.5pt}
\setlength{\abovetopsep}{4pt}

\usepackage[nottoc,numbib]{tocbibind}
\usepackage{physics}

\usepackage[cmintegrals,cmbraces]{newtxmath}
\usepackage{ebgaramond-maths}

\newcommand{\C}{\mathbb{C}}
\newcommand{\R}{\mathbb{R}}
\newcommand{\N}{\mathbb{N}}
\newcommand{\Z}{\mathbb{Z}}

\DeclareMathOperator\erfc{erfc}
\newcommand*\diff{\mathop{}\!\mathrm{d}}
\newcommand*\Diff[1]{\mathop{}\!\mathrm{d^#1}}
\renewcommand{\leq}{\leqslant}
\renewcommand{\geq}{\geqslant}
\newcommand* \e{\mathrm{e}}

\makeatletter
  \DeclareSymbolFont{ntxletters}{OML}{ntxmi}{m}{it}
  \SetSymbolFont{ntxletters}{bold}{OML}{ntxmi}{b}{it}
  \re@DeclareMathSymbol{\leftharpoonup}{\mathrel}{ntxletters}{"28}
  \re@DeclareMathSymbol{\leftharpoondown}{\mathrel}{ntxletters}{"29}
  \re@DeclareMathSymbol{\rightharpoonup}{\mathrel}{ntxletters}{"2A}
  \re@DeclareMathSymbol{\rightharpoondown}{\mathrel}{ntxletters}{"2B}
  \re@DeclareMathSymbol{\triangleleft}{\mathbin}{ntxletters}{"2F}
  \re@DeclareMathSymbol{\triangleright}{\mathbin}{ntxletters}{"2E}
  \re@DeclareMathSymbol{\partial}{\mathord}{ntxletters}{"40}
  \re@DeclareMathSymbol{\flat}{\mathord}{ntxletters}{"5B}
  \re@DeclareMathSymbol{\natural}{\mathord}{ntxletters}{"5C}
  \re@DeclareMathSymbol{\star}{\mathbin}{ntxletters}{"3F}
  \re@DeclareMathSymbol{\smile}{\mathrel}{ntxletters}{"5E}
  \re@DeclareMathSymbol{\frown}{\mathrel}{ntxletters}{"5F}
  \re@DeclareMathSymbol{\sharp}{\mathord}{ntxletters}{"5D}
  \re@DeclareMathAccent{\vec}{\mathord}{ntxletters}{"7E}
\makeatother

\DeclareMathAlphabet\mathbfcal{OMS}{cmsy}{b}{n}

\begin{document}

\title{
  Equação do calor e o problema da adega
}
\author{Caio Tomás de Paula}
\affiliation{Departamento de Matemática, Universidade de Brasília, Brasil.}
%
\begin{abstract}
    Relatório entregue como parte do trabalho final do curso de Introdução a
    Métodos Computacionais em Equações Diferenciais Parciais (IMCEDP) do
    Programa de Pós-Graduação em Matemática (PPGMAT),
    ministrado pelo prof. Dr.~Yuri Dumaresq Sobral no segundo semestre letivo
    de 2023 da Universidade de Brasília.
    O objetivo do trabalho foi resolver, numericamente, a equação do calor.
    Foi utilizada~\cite{lin1998} como referência principal para o trabalho, além
    das notas de aula do curso.
\end{abstract}
%
\maketitle
%
\section{Introdução}
%
	Estamos interessados em estudar a variação da temperatura do solo terrestre a uma dada
	profundidade $x$ no instante $t$. Desconsiderando a curvatura da terra e a variação diária
	de temperatura da superfície, podemos modelar a distribuição de temperatura $u(x,t)$ à
	profundidade $x$ no tempo $t$ por uma equação do calor unidimensional:
	%
	\begin{equation*}
		u_t = \kappa u_{xx}.
	\end{equation*}
	%
	A difusividade térmica do solo terrestre será considerada $\kappa = 6.3\text{m}^2/\text{ano}$.
	Note que estamos desconsiderando o calor oriundo do núcleo da Terra, já que não há forçamento
	na equação. Vamos assumir que $u \xrightarrow{x\to\infty} 0$. Vamos também assumir que
	a temperatura $f(t)$ na superfície ($x = 0$) assume apenas dois valores: uma temperatura
	de ``verão'' durante metade do ano e uma temperatura de ``inverno'' durante a outra metade.
	Esse padrão se repete todo ano, i.e., a temperatura é periódica em $x=0$ com período de 1 ano.

	Em símbolos,
	%
	\begin{equation*}
		f(t) = \begin{cases}
			T_s, \ 0 \leq t < 1/2 \\
			T_w, \ 1/2 \leq t \leq 1,
		\end{cases}
	\end{equation*}
	%
	onde $T_s$ denota a temperatura no verão e $T_w$ denota a temperatura no inverno, com $t$ em anos.

	Vamos tomar a condição inicial $u(x,0) = f(t)\e^{-q_1x}$, com $q_1 = 0.71\text{m}^{-1}$.
	Por fim, vamos tomar $u(L,t) = 0$, com $L$ suficientemente longo para que essa condição seja
	válida.

	Em resumo, queremos resolver o PVIC
	%
	\begin{equation*}
		\begin{cases}
			u_t = \kappa u_{xx}, \ 0 \leq x \leq L, t\geq 0 \\
			u(x,0) = f(t)\e^{-q_1x} \\
			u(0,t) = f(t) \\
			u(L,t) = 0
		\end{cases}
	\end{equation*}
	%

	Queremos resolver este problema numericamente e responder a algumas perguntas.
	Mais especificamente, vamos resolver este problema usando tanto o método de Euler
	explícito (e mostrar sua ordem) quanto o método de Crank-Nicolson, encontrar
	a profundidade ideal para uma adega de vinhos e resolver uma variação
	do problema com difusividade variável.

	Antes de partir para a solução numérica, vamos fazer algumas considerações sobre
	a solução analítica do problema. Primeiro, note que podemos escrever
	%
	\begin{equation}
		f(t) = \sum_{n=-\infty}^{\infty} c_n \e^{2\pi int/T},
	\end{equation}
	%
	com $C_n\in\mathbb{C}$ tais que $\overline{C_n} = C_{-n}$ e $T = 1$ ano.
	É possível mostrar\footnote{~\cite[p.~129]{lin1998}} que, em um tempo $t$,
	uma substância se difunde aproximadamente $\sqrt{\kappa t}$ unidades.
	Sendo assim, na escala de tempo de interesse (1 ano), temos
	%
	\begin{equation}
		\sqrt{\kappa T} = \sqrt{6.3} \approx 2.5\text{ m}.
	\end{equation}
	%
	Vamos tomar o \textit{ansatz}
	%
	\begin{equation}
		u(x,t) = \sum_{n=-\infty}^{\infty} C_n \omega_n(x) \e^{2\pi int/T},
	\end{equation}
	%
	com $C_n\in\mathbb{C}$ tais que $\overline{C_n} = C_{-n}$, de modo que $u$ é real.
	Vamos impor as seguintes condições:
	%
	\begin{itemize}[(i)]
		\item cada uma das parcelas satisfaz a equação do calor
		\item $\omega_n(0) \equiv 1$, de modo que a representação de $f(t)$ seja recuperada
		\item $\omega_n(x)$ é limitado e tende a 0 quando $x\to\infty$ ($n\neq 0$),
		uma vez que a temperatura a profundidades muito grandes não é sensível a variações
		de temperatura na superfície.
	\end{itemize}
	%
	A condição (i) nos dá
	%
	\begin{equation}
		\frac{\diff^2\omega_n}{\diff x^2} = p^{2}_n\omega_n, \ p^{2}_n = \frac{2\pi i n}{\kappa T}.
	\end{equation}
	%
	Consequentemente,
	%
	\begin{equation}
		p_n = \pm(1 \pm i)q_n, \ \text{ com } \ q_n = \sqrt{ \frac{|n|\pi}{\kappa T} } > 0
	\end{equation}
	%
	e o sinal $\pm$ depende do sinal de $n$. Desta forma, a solução geral da EDO de $\omega_n$
	é dada por
	%
	\begin{equation}
		\omega_n(x) = A_n\e^{(1 \pm i)q_n x} + B_n\e^{-(1 \pm i)q_n x}.
	\end{equation}
	%
	Da condição (iii) segue que $A_n \equiv 0$ e da condição (ii) segue que
	$B_n \equiv 1$. Portanto, obtemos
	%
	\begin{equation}
		\omega_n(x) = \e^{-(1 \pm i)q_n x}
	\end{equation}
	%
	e, assim,
	%
	\begin{equation}
		u(x,t) = \sum_{n=-\infty}^{\infty} C_n \e^{-(1 \pm i)q_n x} \e^{2\pi int/T}.
	\end{equation}
	%
	Escrevendo $C_n = |C_n|\e^{-i\gamma_n}$, segue que
	%
	\begin{equation}
		u(x,t) = C_0 + 2\sum_{n=1}^{\infty} |C_n| \e^{-q_n x}\cos\left( \frac{2\pi nt}{T} + \gamma_n - q_n x \right).
	\end{equation}
	%
	Vamos interpretar esta solução. Note que o termo do cosseno representa uma onda de frequência
	$2\pi n/T$ e número de onda $q_n$. Por conta disto, a $n$-ésima ``onda parcial'' se propaga
	com velocidade
	%
	\begin{equation}
		\frac{2\pi n}{q_n T} = \sqrt{ \frac{ 4\pi\kappa|n| }{T} }.
	\end{equation}
	%
	Como há um amortecimento exponencial na direção de propagação e este amortecimento
	cresce com $\sqrt{|n|}$, segue que a contribuição mais importante para a solução
	vem do termo com $n=1$. Para os valores numéricos que estamos considerando,
	segue que
	%
	\begin{equation}
		q_1 = \frac{1}{2\pi}\sqrt{\frac{4\pi\kappa}{T}} \approx 0.71\text{m}^{-1}.
	\end{equation}
	%
	Agora, note que o ponto $x_1$ tal que $q_1x_1 = \pi$ é tal que a temperatura
	$u(x_1,t)$ é oposta em fase à temperatura $u(0,t)$. Dito de outro modo,
	$x_1 = \pi/q_1 \approx 4.4$m é a profundidade na qual a temperatura do solo
	tem fase oposta à temperatura da superfície. Isto significa que se na superfície
	a temperatura é de verão então à profundidade $x_1$ a temperatura é de inverno.
	Além disso, a variação de temperatura é $\e^{-\pi} \approx 4\%$ da variação na superfície.
	Isto torna esta profundidade ideal para uma adega de vinhos!


	%
	\section{Métodos numéricos}
	%

		Começamos resolvendo a equação com o método de Euler explícito, que converge
		para $\kappa\mu \leq 1/2$. Por conta desta restrição ao número de Courant,
		não podemos escolher $\Delta t$ e $\Delta x$ de forma displicente.

		Também utilizamos o método de Crank-Nicolson para resolver o problema.
		Este método, dado por
		%
		\begin{equation}
			\label{eq:crank-nicolson}
			-\alpha u^{n+1}_{\ell + 1} + (1 + 2\alpha)u^{n+1}_{\ell} - \alpha u^{n+1}_{\ell - 1}
			= \alpha u^{n}_{\ell + 1} + (1 - 2\alpha)u^{n}_{\ell} + \alpha u^{n}_{\ell - 1}
		\end{equation}
		%
		com $\alpha = \kappa\mu/2$ é implícito, incondicionalmente estável e consistente
		(logo convergente pelo Teorema de Equivalência de Lax).
		Consequentemente, temos maior liberdade na escolha de $\Delta t$ e $\Delta x$, o que torna
		este método mais indicado para o cômputo da solução por grandes intervalos de tempo.
		A implicitude do método nos obriga a resolver um sistema linear a cada iteração.
		Para tanto, usamos o \href{https://en.wikipedia.org/wiki/Tridiagonal_matrix_algorithm}{algoritmo da matriz tridiagonal}
		(ou de Thomas) para resolver o sistema.

%
\section{Códigos utilizados}
%
	Este projeto pode ser encontrado
	\href{https://github.com/CaioTomas/Trabalho-IMCEDP}{neste link},
	incluindo os códigos utilizados para o cômputo da solução e o código-fonte
	deste documento.


% \printbibliography
\bibliography{Report/referencias}

\end{document}